%%%%%%%%%%%%%%%%%%%%%%%%%%%%%%%%%%%%%%%%%%%%%%%%%%%%%%%%%%%%%
% ROS Project Report Template
% Designed for 2-3 pages (excluding code appendix)
%%%%%%%%%%%%%%%%%%%%%%%%%%%%%%%%%%%%%%%%%%%%%%%%%%%%%%%%%%%%%

\documentclass[11pt, a4paper]{article}

%% ==================== Packages ====================
\usepackage[margin=2.5cm]{geometry}
\usepackage{graphicx}
\usepackage{booktabs}
\usepackage{array}
\usepackage{amsmath, amssymb}
\usepackage{enumitem}
\usepackage{hyperref}
\usepackage{xcolor}
\usepackage{listings}
\usepackage{fancyhdr}
\usepackage{titlesec}
\usepackage{caption}	
\usepackage{subcaption}
\usepackage{float}

%% ==================== Page Style ====================
\pagestyle{fancy}
\fancyhf{}
\fancyhead[L]{\small ROS Project Report}
\fancyhead[R]{\small \thepage}
\renewcommand{\headrulewidth}{0.4pt}

%% ==================== Code Listing Style ====================
\definecolor{codegreen}{rgb}{0,0.6,0}
\definecolor{codegray}{rgb}{0.5,0.5,0.5}
\definecolor{codepurple}{rgb}{0.58,0,0.82}
\definecolor{backcolour}{rgb}{0.95,0.95,0.95}

\lstdefinestyle{mystyle}{
	backgroundcolor=\color{backcolour},   
	commentstyle=\color{codegreen},
	keywordstyle=\color{blue},
	numberstyle=\tiny\color{codegray},
	stringstyle=\color{codepurple},
	basicstyle=\ttfamily\footnotesize,
	breakatwhitespace=false,         
	breaklines=true,                 
	captionpos=b,                    
	keepspaces=true,                 
	numbers=left,                    
	numbersep=5pt,                  
	showspaces=false,                
	showstringspaces=false,
	showtabs=false,                  
	tabsize=2,
	frame=single
}
\lstset{style=mystyle}

%% ==================== Title Section ====================
\title{
	\vspace{-1cm}
	\textbf{[Project Title: e.g., Autonomous Object Tracking Robot]} \\
	\large  Project Report
}

\author{
	Team Member 1 (ID: 20xxxxxx) \and
	Team Member 2 (ID: 20xxxxxx) \and
	Team Member 3 (ID: 20xxxxxx) \and
	Team Member 4 (ID: 20xxxxxx) \and
	Team Member 5 (ID: 20xxxxxx)
}

\date{\today}

%%%%%%%%%%%%%%%%%%%%%%%%%%%%%%%%%%%%%%%%%%%%%%%%%%%%%%%%%%%%%
\begin{document}
	
	\maketitle
	
	%% ==================== 1. Project Overview ====================
	\section{Project Overview}
	
	\subsection{Project Title and Task Description}
	% Briefly describe what the project is about
	This project aims to develop a ROS-based autonomous mobile robot capable of [describe the main task, e.g., tracking and following colored objects in real-time using RGB-D camera].
	
	\subsection{Project Objectives}
	\begin{itemize}[noitemsep]
		\item Objective 1: [e.g., Implement robust color-based object detection]
		\item Objective 2: [e.g., Develop smooth trajectory planning algorithm]
		\item Objective 3: [e.g., Achieve real-time visual servoing control]
	\end{itemize}
	
	
	%% ==================== 2. System Design ====================
	\section{System Design}
	
	\subsection{System Architecture}
	% Include a system architecture diagram
	\begin{figure}[H]
		\centering
		% \includegraphics[width=0.85\textwidth]{figures/system_architecture.png}
		\fbox{\parbox{0.8\textwidth}{\centering\vspace{2cm}[System Architecture Diagram Placeholder]\vspace{2cm}}}
		\caption{System architecture showing ROS nodes and their communication relationships.}
		\label{fig:architecture}
	\end{figure}
	
	\subsection{Module Division}
	The system consists of the following modules:
	\begin{itemize}[noitemsep]
		\item \textbf{Perception Module}: Processes camera data for object detection and localization.
		\item \textbf{Planning Module}: Generates optimal paths based on detected object positions.
		\item \textbf{Control Module}: Executes velocity commands for smooth robot motion.
		\item \textbf{Integration Module}: Coordinates all subsystems and handles exceptions.
	\end{itemize}
	
	% \subsection{Data Flow}
	% Include a data flow diagram
	%\begin{figure}[H]
	%	\centering
		% \includegraphics[width=0.75\textwidth]{figures/data_flow.png}
	%	\fbox{\parbox{0.75\textwidth}{\centering\vspace{1.5cm}[Data Flow Diagram Placeholder]\vspace{1.5cm}}}
		%\caption{Data flow from sensors through processing nodes to actuators.}
	%	\label{fig:dataflow}
	%\end{figure}
	
	
	%% ==================== 3. Algorithm Principles ====================
	\section{Algorithm Principles}
	
	\subsection{Core Algorithm Description}
	% Describe the mathematical principles of your algorithm
	The object detection algorithm utilizes HSV color space segmentation. Given an RGB image \( I \), we first convert it to HSV space:
	\[
	(H, S, V) = f_{RGB \to HSV}(R, G, B)
	\]
	The target region is extracted by thresholding:
	\[
	M(x,y) = 
	\begin{cases}
		1, & \text{if } H_{min} \leq H \leq H_{max} \land S_{min} \leq S \leq S_{max} \\
		0, & \text{otherwise}
	\end{cases}
	\]
	
	\subsection{Algorithm Selection Rationale}
	\begin{table}[H]
		\centering
		\caption{Comparison of Detection Algorithms}
		\begin{tabular}{lccc}
			\toprule
			\textbf{Method} & \textbf{Speed} & \textbf{Accuracy} & \textbf{Complexity} \\
			\midrule
			HSV Thresholding & Fast & Medium & Low \\
			Template Matching & Slow & Low & Medium \\
			\bottomrule
		\end{tabular}
	\end{table}
	
	We selected HSV thresholding because [explain your reasoning].
	
	\subsection{Key Parameters}
	\begin{itemize}[noitemsep]
		\item HSV Range: \( H \in [h_1, h_2] \), \( S \in [s_1, s_2] \), \( V \in [v_1, v_2] \)
		\item Control Gain: \( K_p = 0.5 \), \( K_d = 0.1 \)
		\item Maximum Velocity: \( v_{max} = 0.3 \, \text{m/s} \)
	\end{itemize}
	
	
	%% ==================== 4. Implementation Details ====================
	\section{Implementation Details}

	\subsection{Technical Challenges and Solutions}
	\begin{enumerate}[noitemsep]
		\item \textbf{Challenge}: Noisy depth data causing unstable distance estimation. \\
		\textbf{Solution}: Applied median filter and temporal smoothing.
		\item \textbf{Challenge}: Robot oscillation during tracking. \\
		\textbf{Solution}: Implemented velocity ramping and dead zone.
	\end{enumerate}
	
	\subsection{Debugging Notes}
	% Document issues encountered during debugging
	\begin{itemize}[noitemsep]
		\item Issue: TF frame mismatch between camera and base\_link. Fixed by adding static transform publisher.
		\item Issue: Message queue overflow. Resolved by setting queue\_size=1 for real-time topics.
	\end{itemize}
	
	
	%% ==================== 5. Experimental Results ====================
	\section{Experimental Results}
	
	\subsection{Quantitative Results}
	
	Path planning evaluation encompasses three main categories of metrics.
	
	\textbf{Path Quality Metrics} assess the generated path's characteristics including length (total distance using Euclidean or Manhattan measurements), smoothness (evaluating turns and curvature), safety (obstacle clearance and collision risk), and time efficiency.
	
	\textbf{Algorithm Performance Metrics} measure computational efficiency (runtime, memory, node expansions), solution quality (success rate, optimality ratio), and robustness across varying conditions.
	
	\textbf{Dynamic Environment Metrics} evaluate the algorithm's adaptability through replanning frequency and real-time performance capabilities.
	
	\subsubsection{Path Quality Metrics}
	\begin{itemize}
		\item \textbf{Path Length:} The total distance traveled from start to goal, measured either as straight-line (Euclidean) or grid-based (Manhattan) distance.
		\item \textbf{Smoothness:} The continuity and fluidity of the path, quantified by the number of direction changes, curvature variations, and turning angles.
		\item \textbf{Safety:} The degree of collision avoidance, measured by minimum clearance from obstacles, probability of collision, and safety buffer zones.
		\item \textbf{Time Efficiency:} The temporal performance of path execution, including total travel duration and expected arrival time.
	\end{itemize}
	
	\subsubsection{Algorithm Performance Metrics}
	\begin{itemize}
		\item \textbf{Computational Efficiency:} Resource consumption during path computation, including processing time, memory footprint, and number of search nodes explored.
		\item \textbf{Success Rate:} The percentage of scenarios where the algorithm successfully finds a valid path to the goal.
		\item \textbf{Suboptimality Ratio:} The ratio comparing the found path's cost to the theoretical optimal path cost (value of 1.0 = optimal).
		\item \textbf{Robustness:} The algorithm's stability and consistent performance across diverse environments and varying parameter settings.
	\end{itemize}
	
	\subsubsection{Dynamic Environment Metrics}
	\begin{itemize}
		\item \textbf{Replanning Frequency:} How often the algorithm must recalculate the path due to environmental changes or new obstacles.
		\item \textbf{Adaptability:} The algorithm's ability to adjust to dynamic obstacles, moving targets, or changing constraints.
		\item \textbf{Real-time Performance:} The capability to compute and update paths within strict time constraints for time-critical applications.
	\end{itemize}
	
	\subsection{Quantitative Results}
	\begin{table}[H]
		\centering
		\caption{Performance Metrics}
		\begin{tabular}{lc}
			\toprule
			\textbf{Metric} & \textbf{Value} \\
			\midrule
			Detection Success Rate & 95.3\% \\
			Average Response Time & 42 ms \\
			Tracking Accuracy (RMSE) & 3.2 cm \\
			Average Task Completion Time & 28.5 s \\
			\bottomrule
		\end{tabular}
	\end{table}
	
	\subsection{Qualitative Results}
	\begin{figure}[H]
		\centering
		\begin{subfigure}[b]{0.45\textwidth}
			% \includegraphics[width=\textwidth]{figures/detection_result.png}
			\fbox{\parbox{\textwidth}{\centering\vspace{1.5cm}[Detection Screenshot]\vspace{1.5cm}}}
			\caption{Object detection visualization}
		\end{subfigure}
		\hfill
		\begin{subfigure}[b]{0.45\textwidth}
			% \includegraphics[width=\textwidth]{figures/trajectory.png}
			\fbox{\parbox{\textwidth}{\centering\vspace{1.5cm}[Trajectory Plot]\vspace{1.5cm}}}
			\caption{Robot trajectory during tracking}
		\end{subfigure}
		\caption{Experimental visualization results.}
		\label{fig:results}
	\end{figure}
	
	\subsection{Result Analysis}
	The system achieved [good/satisfactory] performance because [analysis]. However, performance degraded under [conditions] due to [reasons]. Potential improvements include:
	\begin{itemize}[noitemsep]
		\item Implementing adaptive thresholding for varying lighting conditions
		\item Adding Kalman filter for smoother trajectory estimation
	\end{itemize}
	
	
	%% ==================== 6. Team Contribution ====================
	\section{Team Contribution}
	
	\begin{table}[H]
		\centering
		\caption{Team Member Contributions}
		\begin{tabular}{|l|l|l|p{4.5cm}|c|}
			\hline
			\textbf{Name} & \textbf{Student ID} & \textbf{Module} & \textbf{Specific Work} & \textbf{Contribution} \\
			\hline
			Member 1 & 20xxxxxx & Perception & HSV tuning, depth fusion & 30\% \\
			\hline
			Member 2 & 20xxxxxx & Planning & A* algorithm implementation & 25\% \\
			\hline
			Member 3 & 20xxxxxx & Control & Visual servoing, velocity smoothing & 25\% \\
			\hline
			Member 4 & 20xxxxxx & Integration & System debugging, video recording & 20\% \\
			\hline
			Member 5 & 20xxxxxx & ... & ... & ...\% \\
			\hline
			\multicolumn{4}{|r|}{\textbf{Total}} & \textbf{100\%} \\
			\hline
		\end{tabular}
	\end{table}
	
	\vspace{0.5cm}
	\noindent\textbf{Member Signatures:}
	
	\vspace{0.8cm}
	\noindent
	\begin{tabular}{p{3cm}p{3cm}p{3cm}p{3cm}p{3cm}}
		\hrulefill & \hrulefill & \hrulefill & \hrulefill & \hrulefill \\
		Member 1 & Member 2 & Member 3 & Member 4  & Member 5 \\
	\end{tabular}
	
	
	%% ==================== 7. Summary and Reflection ====================
	\section{Summary and Reflection}
	
	\subsection{Project Achievements}
	\begin{itemize}[noitemsep]
		\item Successfully implemented a complete ROS-based robotic system
		\item Gained hands-on experience with computer vision and robot control
		\item Developed teamwork and project management skills
		\item Deepened understanding of sensor fusion and real-time systems
	\end{itemize}
	
	
	%% ==================== Appendix (Optional) ====================
	\appendix
	\section{Appendix}
	
	\subsection{Code Repository}
	Complete source code is available at: \url{https://github.com/username/project-repo}
	
	\subsection{Additional Experimental Data}
	% Add any extra data tables or figures here
	
	\subsection{References}
	\begin{enumerate}[noitemsep]
		\item ROS Wiki: \url{http://wiki.ros.org/}
		\item OpenCV Documentation: \url{https://docs.opencv.org/}
		\item Additional references as needed
	\end{enumerate}
	
\end{document}
